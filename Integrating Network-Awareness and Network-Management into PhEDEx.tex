%\documentclass[toc]{Template/PoS}
\documentclass{Template/PoS}

\usepackage{enumitem}
\setlist{nolistsep}

\title{Integrating Network-Awareness and Network-Management into PhEDEx}

\ShortTitle{Network-Management in PhEDEx}

\author{\speaker{Vlad L\v{a}p\v{a}d\v{a}tescu}\\
Caltech / USA\\
E-mail: \email{vlad@cern.ch}}

\author{Tony Wildish\\
Princeton / USA\\
E-mail: \email{awildish@princeton.edu}}

\abstract{
PhEDEx is the data placement and management tool for the CMS experiment at the LHC. It controls the large-scale data-flows on the WAN across the experiment, typically handling 1 PB of data per week.
While robust, its architecture is now ten years old and has yet to fully
adapt to today's production environment, an environment in which the network is 
the fastest and most reliable component.

The ANSE (Advanced Network Services for Experiments) project, in the context of 
CMS, aims to greatly improve PhEDEx' network awareness for smart source 
selection, as well as to integrate bandwidth provisioning capabilities in the data 
transfer management. Both parts require a good knowledge of the network status, 
topology and of course, access to useful and up-to-date performance metrics.

One of the first steps towards this goal involved the identification of a mechanism 
for informing PhEDEx about independent network performance metrics. Methods 
for providing these metrics have been prototyped and verified in a LAN testbed 
using fake data transfer requests. This mechanism is already directly usable by CMS 
in their production environment.

Currently, the ANSE-PhEDEx testbed is spread over many servers at a number of 
sites. It is composed of several machines dedicated to PhEDEx site agents, one 
server holding the PhEDEx central agents, a central database and one server which 
contains the PhEDEx website and data-service. Some of the site nodes have 
additional attached storage nodes.

In this paper, we present the work that has been done in ANSE for PhEDEx.
This includes performance measurements using
the Fast Data Transfer (FDT) tool and the extension of the PhEDEx agent that downloads files to a site
to allow it to control the network via creation and use of dynamic circuits. We
present the results of our tests using these new features, on high-speed 
WAN circuits ranging from a few Gbps to 40Gbps and detail the development 
done within PhEDEx itself.

Finally, the paper will also describe the future plans for the project.}

\FullConference{International Symposium on Grids and Clouds (ISGC) 2014,\\
		23-28 March 2014\\
		Academia Sinica, Taipei, Taiwan}

\parindent=0pt 

\begin{document}


%TODO% 
%- Create subsection explaining why the average rates in PhEDEx are lower than average job transfer rates.
    %This is written in  ''Integration at the FileDownload Level'' but is common to ''Integration at the transfer-level''  

    
%- Tidy up section numbering 

    
\section{Introduction}
PhEDEx \cite{PhEDEx} is the data-placement management tool for the CMS \cite{CMS} experiment at 
the LHC. It manages the scheduling of all large-scale WAN transfers in CMS, 
ensuring reliable delivery of the data. It consists of several components:

\begin{itemize}
\item an Oracle database, hosted at CERN
\item a website and data-service, which users (humans or machine) use to interact with and control 
PhEDEx
\item a set of 'central' agents that deal with routing, request-management, bookeeping and other 
activities. These agents are also hosted at CERN, though they could be run anywhere. The 
key point is that there is only one set of central agents per PhEDEx instance
\item a set of 'site-agents', one set for every site that receives data
\end{itemize}

PhEDEx maintains knowledge and history of transfer 
performance, and the central agents use that information to choose among source replicas when a 
user makes a request. The central agents then queue the transfer to be processed by the site 
agents. PhEDEx operates in a data-pull mode, the destination site pulls the data to itself when it 
is ready. This gives the sites more control over the activity at their site, so they can ensure 
that neither their network nor their storage is overloaded.

PhEDEx was originaly conceived over ten years ago now, and the architecture still reflects design 
decisions made at that time. Then, the network was expected to be the weakest link in 
the developing Worldwide LHC Grid (WLCG) \cite{WLCG}. Networks were expected to have bandwidth of 
the order of 100 Mb/sec, to be unreliable, and to be poorly connected across the span of the CMS 
experiment. Accordingly, PhEDEx will back off fast and retry gently in the face of failed 
transfers, on the assumption that failures will take time to fix, and that there is other useful 
data that can use the network in the meantime.

The data-transfer topology was designed with a 
strongly hierarchical structure, with the Tier-0 (CERN) transferring data primarily to a set of 
6-7 Tier-1 sites, and each Tier-1 site handling traffic between itself, the other Tier-1s. and 
it's local Tier-2 sites. This kept the transfer-links (i.e. the set of (source,destination) pairs) 
mostly in the realm of a single regional network operator, and kept the overall number of transfer 
links low. Transfers between arbitrary pairs of sites were not allowed.

Today, the reality is very different. The network has emerged as the most reliable component.

\section{ANSE testbed}

In the context of the ANSE project, the central agents, DB and frontend all run
at CERN. The site agents are running on machines located in different geographical
locations. (see Figure \ref{fig:ANSE-setup}).

\begin{figure}[h]
  \centering
  \includegraphics[width=0.95\textwidth]{Figures/ANSE-setup.png}
  \caption{PhEDEx setup for ANSE}
  \label{fig:ANSE-setup}
\end{figure}

For development work and prototype tests we mainly use OpenStack virtual machines.
These contain PhEDEx installations and also double as small storage nodes. These
typically have 8GB of RAM, 4 VCPUs and 80GB disk. These VMs are able to have
sustained disk-to-disk transfer rates of 1Gbps.

For more advanced tests we use four physical servers, two located in Geneva and 
two in Amsterdam. In each location, one of the two servers is used as a PhEDEx node, 
while the other is used as a storage node.

Each storage node (sandy01-gva and sandy01-ams) has dual E5-2670 CPUs (32 logical cores)
64GB or RAM and 2 LSI disk controllers. Each of the LSI controllers manages
8 high speed SSDs. Two RAID-0 arrays are created on each LSI controller (4 SSD per
RAID 0 array).

Between the two storage nodes we have a high capacity 40Gbps link and use Ciena
routers to create dynamic links between them.

Since the storage nodes were designed for high speed rather than high capacity,
we created a large number of soft links to large random-filled files (2->15GB) 
on the source disk. 

On the destination side we either used a crontab running every minute to remove
transferred files, or we relied on the PhEDEx post-validation  script to remove 
files in bulk at the end of a transfer job.

Monitoring is performed with MonALISA\cite{MonALISA} and PhEDEx.

\section{Integrating external sources of information}

\subsection{Sources of information}

Currently CERN runs three PhEDEx instances:

\begin{description}
	\item[Production] is used for production CMS data transfers
	\item[Debug] is used for link monitoring and link commissioning tests
	\item[Development] is used for testing software and DB schema upgrades
\end{description}

Although the instances run on the same infrastructure, each one is independent 
of the rest. The Production instance choses transfer sources based on its 
internal monitoring data alone. The Debug instance on the other hand,
has monitoring information about many more links. Because of this, it makes
sense that where links are shared, network statistics should be shared
as well (Figure \ref{fig:Prod-vs-Debug})

\begin{figure}[h]
  \centering
  \includegraphics[width=0.95\textwidth]{Figures/Debug-vs-Prod-instance.png}
  \caption{Example case where the Production and Debug instance contain complementary
  data}
  \label{fig:Prod-vs-Debug}
\end{figure} 

A tool has been developed which retrieves statistics from all instances and 
aggregates the data. This tool then produces a file that the FileRouter agent
can use.

The next step is to integrate it with perfSONAR, effectively gaining a 
more comprehensive picture of what's happening on the network. At the time of
writing this paper, perfSONAR is yet to be deployed on all sites. We also
lack a perfSONAR API which we can call to retrieve this data.
\section{Integrating virtual circuts into PhEDEx}
There are several points in the PhEDEx software stack at which it is possible to integrate the control and use of virtual circuits; per transfer-job, at the level of the FileDownload agent, and at the FileRouter agent.

At the lowest level, a circuit can be requested per transfer-job, by the transfer tool called from the FileDownload agent. This level of integration was already achieved some time ago with the initial implementation of the FDT backend for the FileDownload agent. Each transfer-job would attempt to create its own virtual circuit, if it failed it would simply proceed to transfer over the general-purpose network.

The advantage of this approach is simplicity, there is nothing to modify in PhEDEx itself, all the work happens in the external transfer tool. There are, however, a few disadvantages:
\begin{itemize}
\item Transfer-jobs tend to be short-lived (tens of minutes rather than hours), so circuits are being created at a high rate. This is not currently sustainable.
\item PhEDEx is typically configured to overlap transfer-jobs, to offset the overheads incurred between successive jobs. This means that concurrent transfer-jobs will be competing for circuits, with consequences that may be hard to understand. For example, if two jobs can share a circuit, what happens to one of them when the other finishes, and tears down the circuit?
\item There is no overview of the use of network resources. Circuits are requested based on nothing more than the local knowledge of a single link, without regard for other traffic flows into the destination or out of the source.
\end{itemize}

The next level of integration is at the level of the FileDownload agent. This site-agent has an overview of the data-flows into the destination site at any time, so can make intelligent decisions about creating circuits. Apart from simply deciding that it should - or should not - create a circuit for a given flow, it can also create circuits that are longer-lived, serving a series of transfer-jobs lasting several hours. This will make for much more stable operation, both from the viewpoint of the network fabric and from the viewpoint of the transfer-jobs. This is the first level of integration at which virtual circuit control becomes really practical for production use by PhEDEx. The only significant disadvantage is that destination sites still make circuit-requests without regard for conditions at the source sites, so the risk of sub-optimal configurations still exists. For example, site A may book a circuit to site B to pull data from it at high speed, but that may commit most of site B's WAN bandwidth, leaving little for site B to download from sites C, D and E.

The highest level of integration is at the level of the FileRouter, or centrally, with a view of the entire transfer system. The FileRouter is essentially a high-level scheduler, deciding the set of source-destination pairs for fulfilling transfer requests and building the queues of transfers for each destination site. As such, it alone can have an overview of how busy each site is expected to be, so it could decide which transfers should use circuits and which should not. It could even base its choice of source-replica for a given transfer on availability of circuits and bandwidth, not just on current network performance between the sites. It can also decide that some transfers are low priority, so should not request a circuit even if one might be available, in order to leave that resource free for higher priority requests that might come along.

There are two disadvantages to integration at the FileRouter. The first is that the FileRouter deals with transfers that \emph{will} happen, in the near future. When something happens to invalidate its predictions (e.g. a power-cut at a site) this will invalidate those predictions to a greater or lesser degree. The more coupled the sites are in their scheduling, the wider the implications of a single site-outage.

The second disadvantage is the complexity of the problem. Developing efficient scheduling algorithms, including predicting the availability of circuits on a shared network, is not a trivial task. For example, queues may be re-organised to take advantage of circuits, instead of just using raw priority and age as the queueing criteria. The FileRouter agent will also have to inform FileDownload agents when they are expected to use a circuit and when not to, and not just which files to transfer. Despite these complications, having centrally managed knowledge of when and where to create circuits for PhEDEx is the ultimate goal.
\section{Integration at the transfer level}

\subsection{Integrating FDT into PhEDEx}
The FDT\cite{FDT}  tool integrates IDCP
\footnote{InterDomain Controller Protocol} (OSCARS)
\footnote{On-Demand Secure Circuits and Advance Reservation System} calls so
integrating it in PhEDEx naturally gives us BoD\footnote{Bandwidth on Demand}
capabilities. In addition to this, we discovered that FDT performs extremely well
as a transfer tool in itself, as we will show in this paper.

As shown in Figure \ref{fig:PhEDEx-FDT-arch}, the current architecture consists
of four main components:
\begin{description}
	\item[FDT tool] - written in Java it's based on an asynchronous, flexible 
multithreaded system, using the capabilities of the Java NIO\footnote{New I/O} 
libraries. It's capable of reading and writing at disk speed over wide area 
networks and runs on all major platforms. 
	\item[PhEDEx backend*] - written in Perl, this backend
receives transfer jobs from the FileDownload site agent which will in turn 
invoke the fdtcp wrapper. 
	\item[Fdtcp wrapper*] - written in Python this is the 
interface between PhEDEx and FDT. It prepares the file list as required by FDT, 
invokes the fdtd service and harvests reports to be propagated back to PhEDEx.
	\item[Fdtcp daemon*] - written in Python as well, it's 
a permanently running daemon, in charge with authentication and completion of requests 
from fdtcp. These requests are transmitted via PYRO\footnote{PYthon Remote Objects} 
calls and will either launch the FDT in client mode on source sites or FDT in server 
mode on destination sites.
\end{description}

The components marked with * were not developed by the same team as the one responsible 
for FDT.

\begin{figure}[h]
  \centering
  \includegraphics[width=0.95\textwidth]{Figures/PhEDEx-and-FDT-diagram.png}
  \caption{Diagram of various components needed to integrate FDT into PhEDEx}
  \label{fig:PhEDEx-FDT-arch}
\end{figure} 

Figure \ref{fig:PhEDEx-FDT-arch} also describes a normal scenario in which Site B
needs to transfer data from Site A. In this example, the PhEDEx instance and the
storage node are separate physical servers. 

Once the FileDownload agent on the PhEDEx server on Site A, marks all the files as
ready for transfer, the FDT backend on that machine will then issue an fdtcp 
command with the list of files that needs to be copied from the storage node of
Site A to the storage node of Site B.

Fdtcp will in turn, invoke the fdtd daemons on the two storage nodes. Each fdtd
daemon will first verify that the command comes from an approved server, after
which it will then launch the appropriate FDT tool. The fdtd daemon on Site B 
will start the FDT tool in server mode and the daemon on site B starts FDT in 
client mode.

Once this is done, the transfers are handled by FDT. After the transfer finishes
the reports are then propagated up the chain, eventually reaching PhEDEx.

\subsection{Setup and transfer results with FDT}

\subsubsection{Setup}

The tests were run on our setup (Figure \ref{fig:ANSE-setup}), using the Geneva
(T2\_ANSE\_Geneva) and Amsterdam (T2\_ANSE\_Amsterdam) sites. All LSI controllers
were used.

Each transfer job was 2.25TB (150x15TB files).

\subsubsection{Transfer results with FDT}

As seen in Figure \ref{fig:FDT-Transfers} and Figure \ref{fig:FDT-Transfers-PhEDEx}
FDT is able to achieve sustained rates of over 1500MB/sec. As PhEDEx shows,
the average sits at a marginally slower 1360MB/sec. As visible in 
Figure \ref{fig:FDT-Transfers}, the plot has periodical drops in transfer rates.
This is due to the fact that between each transfer job (which is composed of a limited
number of files) there is a delay between the ending of one job and the launch of
a new one. This is what accounts for the difference in average reported rates by PhEDEx
and the average rate for each transfer job.

These high transfer rates are not only due to the storage system alone... FDT 
automatically starts the correct number of readers and writers if the files at 
the source and destinations sites are spread on different physical disks/controllers.

\begin{figure}[h]
  \centering
  \includegraphics[width=0.95\textwidth]{Figures/FDT-transfers.png}
  \caption{PhEDEx transfers over 24hrs using FDT - MonALISA plot}
  \label{fig:FDT-Transfers}
\end{figure} 

\begin{figure}[h]
  \centering
  \includegraphics[width=0.95\textwidth]{Figures/FDT-transfers-PhEDEx.png}
  \caption{PhEDEx transfers over 24hrs using FDT - PhEDEx plot (in MB/sec)}
  \label{fig:FDT-Transfers-PhEDEx}
\end{figure} 


\subsubsection{Caveats}

At the moment this system does have some drawbacks:

\begin{itemize}
	\item This setup is not used by anyone in production at the moment
	\item FDT requires a POSIX compatible interface to function. At the moment
not all PhEDEx sites are capable of exposing that.
	\item In order to get the best performance out of the systems, files that
are to be transfered need to be spread among different disks on both source
and destination storage servers. This puts an additional configuration effort
on the sites and adds more complexity in the TFCs\footnote{Transfer File
Catalogs}
\end{itemize}
\section{Virtual circuits at the FileDownload Agent Level}

\subsection{Changes to the FileDownload agent}

For this prototype we decided to integrate all of the control and circuit awareness
logic in the FileDownload agent, thus eliminating the need for an additional site 
agent or any changes to the DB schema. 

To ease development and to facilitate testing, we fake calls to circuit management APIs
which in turn, return IPs of already established static circuits. From PhEDEx'
point of view it's as if a new path has been created. If some conditions are
met, PhEDEx switches to using this path, as it would with a dynamic circuit.

\subsubsection{Making use of a circuit}

PhEDEx transfers files in bulk, copying several files with each transfer command. The number of files varies from
site to site depending on local constraints, but is typically in the range of a few tens of files per transfer job. Each transfer job therefore contains information about the source and destination Physical File Names (PFNs) of several files.

A PFN contains the protocol that needs to be used, the hostname or IP of the file
and the local path of that file. This means that if we want PhEDEx to use
circuits, we need a mechanism to replace the original hostname or IP in each 
source/destination PFN with the source IP and destination IP of the circuit.

The FileDownload agent actually receives only Logical File Names (LFNs) from the database, plus the name of the chosen source site. It constructs the PFNs from the LFNs and a lookup-table per-site which each site maintains and uploads to the database. After calculating the source and destination PFNs, files are bundled into transfer jobs and queued for submission with whichever transfer tool has been configured between the two sites.

When the transfer jobs are taken from the local queue, but before they are actually submitted, the agent will check to see if a circuit has been established between the two endpoints. If so, it manipulates the source and destination PFNs, replacing the hostnames or IP numbers of the endpoints with the IP numbers corresponding to the circuit. This manipulation is completely transparent to the actual transfer tool in use.

It could also happen that a circuit becomes available, while the FileDownload agent
marks tasks ready for transfer. This means that part of the tasks in a job
will contain source/destination PFNs having the original hostname/IPs, while
others will use the circuit IPs. When the FDT backend is used, it will automatically
launch two different jobs: one for the files transferred over the normal path
and the other for files transferred over the circuit. We still need to test
how the other backends react. In any case, even if a transfer job would fail 
because of this, PhEDEx will just retry, as it would for any failure, and the transfer would succeed the second
time.

\subsubsection{Circuit awareness and lifecycle management}

Since the standard FileDownload agent doesn't have knowledge about more than one
transfer paths over a given link we needed to add this circuit awareness and a
way to manage the lifecycle of a circuit on a given link.

PhEDEx agents are event-driven, using the Perl Object Environment (POE\cite{POE}) framework.
To implement circuit-control we added several POE controlled events. Here are the main ones:

\begin{itemize}
  \item check\_workload
  \item request\_circuit
  \item handle\_request
  \item teardown\_circuit
  \item check\_circuit\_setup
\end{itemize}

\paragraph{check\_workload}
This is a recurrent event at 60 second intervals. It is used to estimate the 
amount of work that remains to be done based on the current size of the download
 queue. In order to do this, it needs to know the average rate that the current 
link is capable of and the total number of pending bytes. The latter is calculated
based on the PENDING tasks that it currently holds. The former can be retrieved
in two ways: 

\begin{itemize}
  \item if the agent has recently transfered tasks it will retain information about
previously DONE tasks, which will then be used for the average rate calculation
  \item if nothing has been transferred, it will try to get this information based
on the link average rate calculated by PhEDEx itself.
\end{itemize}

If an average rate source-destination pair has been calculated, it will estimate
the amount of work needed to be done.

Going through all of the links, we check if all of the conditions below had been met:
\begin{itemize}
  \item the amount of work pending is above a given amount\footnote{This is arbitrarily set to five hours}
  \item a circuit has not been established
  \item a circuit request is not pending
  \item a circuit request hasn't failed within a given time-window\footnote{This time-window is arbitrarily set to one hour}
  \item transfers haven't previously failed, within a given time-window, while a circuit was active 
\end{itemize}

If this is the case, we trigger the request\_circuit event for that particular link.

\paragraph{request\_circuit}

As the name suggests this event is used to request a circuit.

It does the following:
\begin{itemize}
  \item creates a state-file for this request. This file contains the names of the PhEDEx nodes involved in the request, the time of the request and the 
lifetime of the circuit. This file is used to maintain knowledge of circuits across restarts of the FileDownload agent
  \item sets a timer to cancel the request if there is no timely response. If we don't
get a circuit within 10 minutes it's very likely that we won't get it at all.
  \item in the case of this prototype the "handle\_request" event is triggerred directly,
however in the production version, this event will actually be a callback
from the API call where a circuit is being requested
\end{itemize}

\paragraph{handle\_request}

This event is a callback from the API call to request a circuit.

It will:
\begin{itemize}
  \item trigger the event "remove\_circuit\_request"
  \item if the circuit creation failed, flag it and return
  \item modify and save the state file corresponding to this request by
adding relevant information concerning the circuit that has just been established:
time the circuit was established, time the circuit expires, endpoint IPs of 
the established circuit.
  \item starts a timer for the "teardown\_circuit" event if an expiration
time has been specified for the circuit
\end{itemize}

\paragraph{check\_circuit\_setup}

This is another recurrent event at 60 second intervals and is used to provide
sanitation in case of errors. There are two important scenarios that we are interested in:
\begin{enumerate}
  \item Download agent has crashed and is restarted
  \begin{enumerate}
    \item In-memory state information is lost, but circuit-request state file(s) exist on disk:\\
			For each circuit-request state file that has no matching in-memory data,
			try to cancel the circuit request and remove the file from disk
    \item In-memory state information is lost, but circuit-established state file(s) exist on disk:\\
			For each circuit-established state file that has no matching in-memory data,
			check the expiration time in the state file:
			\begin{itemize}
			  \item if the expiration time is not defined, or is defined but didn't expire yet,
					  the circuit can be reused. Internal data is populated based on state
					 file and if the expiration time is defined, the teardown\_circuit timer is
					 set.
			  \item if the expiration time is defined at it expired, remove the state file and
					 try to tear down the circuit 
			\end{itemize}
  \end{enumerate}
  \item Reconsider failed circuits with failed requests or transfers:\\
		  Creating a circuit on a particular link may have been blacklisted due to requests
		  for circuits failing or due to files transfers failing on that circuit. After a given
		  time these circuits are removed from the blacklist and the system is free to try
		  to use them again.
\end{enumerate}

\begin{description}
	\item[handle\_request\_timeout] \hfill \\
		Event triggered after a timeout, it's used to cancel request and clean internal state
	\item[remove\_circuit\_request] \hfill \\
		Cleans up internal data and state file concerning this circuit request
	\item[teardown\_circuit] \hfill \\
		Cleans up internal data and state file concerning the circuit and calls the API for 
		tearing down the circuit.
\end{description}

\subsection{Setup and test}

The prototype was tested on ANSE's testbed (Figure \ref{fig:ANSE-setup}) using 
our two PhEDEx sites: \\ T2\_ANSE\_Amsterdam and T2\_ANSE\_Geneva. As we mentioned earlier
each LSI controller manages 8 SSDs split in two RAID 0 arrays. Since the tests are write 
 intensive and were scheduled to run for around 24 hours, we decided to only use 
 1 LSI controller, in order to minimize the negative effects on the life of the SSDs.


Two 10Gbps virtual circuits were created for the purposes of this test. Figure \ref{fig:testbed}
depicts the setup used.

\begin{figure}[h]
  \centering
  \includegraphics[width=0.95\textwidth]{Figures/FileDownload_ANSE_Testbed}
  \caption{Diagram of the testbed used by the prototype}
  \label{fig:testbed}
\end{figure} 

One of the circuits was used to model a shared link in which PhEDEx had to compete 
with other traffic. This background traffic was generated by Iperf and consisted of a 
continuous stream of UDP packets at 5Gbps . The other circuit served as the dedicated
link.

The main purpose of this test wasn't to show that we can saturate a 10Gbps link with
PhEDEX (although we came close with just 1 LSI controller), but that a PhEDEx FileDownload agent 
is able to switch to using a new path in a transparent manner and with no down time.

The first part of the test consisted of a 10 hour run with PhEDEx transfers on the 
shared link. After this time, PhEDEx switched to using the dedicated circuit and
continued transfers for another 10 hours.

We set up PhEDEx to run a single 450 GB transfer job at a time, each one 
comprised of 30 files of 15 GB each.

\subsection{Results}

The results of the first half of the test are shown in  (Figure \ref{fig:shared_transfers}).
A quick glance at this plot, shows that the 10 Gbps link was saturated by the two
competing transfers. Between 23:00 and 0:00 (beginning of the plot) and 
10:00 and 11:00 (end of the plot), only UDP traffic was sent across the network, 
running at a steady 5 Gbps. This effectively leaves only 5 Gbps of PhEDEx traffic.
PhEDEx transfers start around 0:00 and quickly saturates the 10 Gbps. 
The seesaw look of this plot is due to the fact that PhEDEx has a delay
between finishing one job and starting the next. This is due to various factors:
 pre/post validation, preparation of copyjobs or even time spent by the backend itself 
 before actually launching a transfer. Because of these delays, the average 
 rates reported by PhEDEx will always be lower than the average rates of each
 individual transfer job. Nevertheless we get average transfer rates of 9.5 Gbps for the
 whole link and consequently 4.5 Gbps for PhEDEx transfers (10\% penalty from
 gap between jobs).

\begin{figure}[h]
  \centering
  \includegraphics[width=0.95\textwidth]{Figures/FileDownload_Shared_path.png}
  \caption{PhEDEx transfers on the shared path - competing with 5Gbps UDP traffic. As with
  the previous case, the red line represents network throughput in Mbps while the blue
  line represents a moving average of one hours of that network throughput.}
  \label{fig:shared_transfers}
\end{figure} 

Around 10:00 PhEDEx switches to using the dedicated link. The results of the 
second half of the test are shown in Figure \ref{fig:solo_transfers}.

Although most of the time we are able to saturate the 10Gbps link with PhEDEx 
traffic alone we sometimes see dips in transfer rates. These dips can be attributed 
to the storage not being able to sustain such high write rates.

This time, the average link rates drop from 9.5 Gbps to 8.5 Gbps, however 
actual PhEDEx transfers go up from 4.5 Gbps to 8.5 Gbps! The reason for this 
drop in average link rate is two fold: 

\begin{itemize}
  \item Transferring a job at higher speeds means that it will take less time for
  it to complete, hence more jobs will be completed in one hour. However, as
  we previously explained, there is a fixed overhead for starting each new job, which
  means that more delays will be introduced into the system.
  \item The storage system (using 1 LSI controller) sometimes cannot sustain
  write rates of 10 Gbps to disk.
\end{itemize}

\begin{figure}[h]
  \centering
  \includegraphics[width=0.95\textwidth]{Figures/FileDownload_Solo_path.png}
  \caption{PhEDEx transfers on the dedicated path}
    \label{fig:solo_transfers}
\end{figure} 

Figure \ref{fig:combined_transfers} and Figure \ref{fig:combined_phedex_transfers} show
the  network activity for PhEDEx traffic alone on both links. 
In this plot we can observe that PhEDEx switches to using the
new link without any interruption in service, doing it seamlessly and in a 
transparent way for other site-agents (switch occurs around 10:00).

\begin{figure}[h]
  \centering
  \includegraphics[width=0.95\textwidth]{Figures/FileDownload_All_paths.png}
  \caption{View of PhEDEx-only transfers on both the shared and dedicated path}
  \label{fig:combined_transfers}
\end{figure} 

\begin{figure}[h]
  \centering
  \includegraphics[width=0.95\textwidth]{Figures/FileDownload_PhEDEx_all_paths.png}
  \caption{View of PhEDEx only transfers on both the shared and dedicated path}
  \label{fig:combined_phedex_transfers}
\end{figure} 
%% Vlad, this is for you
\begin{thebibliography}{1}

\bibitem{PhEDEx}
 Egeland R, Metson S and Wildish T 2008 Data transfer infrastructure for CMS data taking,  {\it XII Advanced Computing and Analysis Techniques in Physics Research (Erice, Italy: Proceedings of Science)}

\bibitem{CMS}
 The CMS Collaboration 2008 The CMS experiment at the CERN LHC {\it JINST {\bf 3} S08004}

\bibitem{FTS}
 Alvarez Ayllon A, Kamil Simon M, Keeble O and Salichos M 2013 FTS3 - Robust, simplified and high-performance data movement service for WLCG {\it submitted to CHEP 2013}

\bibitem{MonALISA}
 Legrand I, Newman H, Voicu R, Cirstoiu C, Grigoras C, Dobre C, Muraru A, Costan A, Dediu M and Stratan C 2009 MonALISA: An agent based, dynamic service system to monitor, control and optimize distributed systems {\it Computer Physics Communications, Volume 180, Issue 12, December 2009, Pages 2472 - 2498}

\bibitem{MonALISA site}
 MONitoring Agents using a Large Integrated Services Architecture (MonALISA) {\it http://monalisa.caltech.edu/}
 
\bibitem{FDT}
 Fast Data Transfer (FDT) {\it http://fdt.cern.ch/}
 
\bibitem{WLCG}
 Eck C {\it et al.} 2005 LHC Computing Grid Technical Design Report {\it CERN-LHCC-2005-024}

\bibitem{TW_DB_CHEP13}
 Bonacorsi D and Wildish T 2013 Challenging CMS Computing with Network-Aware Systems {\it submitted to CHEP 2013}

\bibitem{ANSE}
 \textbf{ANSE: Need a reference}

\bibitem{perfSONAR}
  Campana S et.al. 2013 Deployment of a WLCG network monitoring infrastructure based on the perfSONAR-PS technology {\it submitted to CHEP 2013}

\bibitem{OSCARS}
  Guok C, Robertson D, Thompson M, Lee J, Tierney B, Johnston W 2006: Intra and Interdomain Circuit Provisioning Using the OSCARS Reservation System {\it 3rd International Conference on Broadband Communications, Networks and Systems, 2006. BROADNETS 2006}

\bibitem{POE}
 The Perl Object Environment (POE) {\it http://poe.perl.org/}

% F.~Baggins, \emph{Quantum effects of the One Ring},
% \emph{JHEP} {\bf 01} (3021) 006 [{\tt hep-th/2001033}].

\end{thebibliography}


\end{document}