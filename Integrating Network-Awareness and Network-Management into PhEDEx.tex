\documentclass[toc]{Template/PoS}

\title{Integrating Network-Awareness and Network-Management into PhEDEx}

\ShortTitle{Network-Management in PhEDEx}

\author{\speaker{Vlad L\v{a}p\v{a}d\v{a}tescu}\\
Caltech, Switzerland\\
E-mail: \email{vlad@cern.ch}}

\author{Tony Wildish\\
Princeton / USA\\
E-mail: \email{awildish@princeton.edu}}

\abstract{
PhEDEx is the data placement and management tool in CMS, which controls bulk 
data-flows. While robust, its architecture is now ten years old and has yet to fully 
adapt to today's production environment, an environment in which the network is 
the fastest and most reliable component.

The ANSE (Advanced Network Services for Experiments) project, in the context of 
CMS, aims to greatly improve PhEDEx' network awareness for smart source 
selection, as well as to integrate bandwidth provisioning capabilities in the data 
transfer management. Both parts require a good knowledge of the network status, 
topology and of course, access to useful and up-to-date performance metrics.

One of the first steps towards this goal involved the identification of a mechanism 
for informing PhEDEx about independent network performance metrics. Methods 
for providing these metrics have been prototyped and verified in a LAN testbed 
using fake data transfer requests. This mechanism is already directly usable by CMS 
in their production environment.

Currently, the ANSE-PhEDEx testbed is spread over many servers at a number of 
sites. It is composed of several machines dedicated to PhEDEx site agents, one 
server holding the PhEDEx central agents, a central database and one server which 
contains the PhEDEx website and data-service. Some of the site nodes have 
additional attached storage nodes.

In this paper, we present the results of PhEDEx data transfers using the Fast Data 
Transfer (FDT) tool, on high-speed WAN circuits ranging from a few Gbps up to 40
 Gbps. We continue by discussing their implications for intelligently controlling the 
network from within PhEDEx. This involves development within PhEDEx itself, 
integration between PhEDEx and various network fabric tools, and possibly 
requirements for development of the fabric tools themselves, to support more
 sophisticated PhEDEx use-cases.

Finally, the paper will also describe the future plans for the project.}

\FullConference{International Symposium for Grids and Computing 2014

Taipei, Taiwan

25-28 March 2014}

\begin{document}

\section{Introduction}
PhEDEx \cite{PhEDEx} is the data-placement management tool for the CMS \cite{CMS} experiment at 
the LHC. It manages the scheduling of all large-scale WAN transfers in CMS, 
ensuring reliable delivery of the data. It consists of several components:

\begin{itemize}
\item an Oracle database, hosted at CERN
\item a website and data-service, which users (humans or machine) use to interact with and control 
PhEDEx
\item a set of 'central' agents that deal with routing, request-management, bookeeping and other 
activities. These agents are also hosted at CERN, though they could be run anywhere. The 
key point is that there is only one set of central agents per PhEDEx instance
\item a set of 'site-agents', one set for every site that receives data
\end{itemize}

PhEDEx maintains knowledge and history of transfer 
performance, and the central agents use that information to choose among source replicas when a 
user makes a request. The central agents then queue the transfer to be processed by the site 
agents. PhEDEx operates in a data-pull mode, the destination site pulls the data to itself when it 
is ready. This gives the sites more control over the activity at their site, so they can ensure 
that neither their network nor their storage is overloaded.

PhEDEx was originaly conceived over ten years ago now, and the architecture still reflects design 
decisions made at that time. Then, the network was expected to be the weakest link in 
the developing Worldwide LHC Grid (WLCG) \cite{WLCG}. Networks were expected to have bandwidth of 
the order of 100 Mb/sec, to be unreliable, and to be poorly connected across the span of the CMS 
experiment. Accordingly, PhEDEx will back off fast and retry gently in the face of failed 
transfers, on the assumption that failures will take time to fix, and that there is other useful 
data that can use the network in the meantime.

The data-transfer topology was designed with a 
strongly hierarchical structure, with the Tier-0 (CERN) transferring data primarily to a set of 
6-7 Tier-1 sites, and each Tier-1 site handling traffic between itself, the other Tier-1s. and 
it's local Tier-2 sites. This kept the transfer-links (i.e. the set of (source,destination) pairs) 
mostly in the realm of a single regional network operator, and kept the overall number of transfer 
links low. Transfers between arbitrary pairs of sites were not allowed.

Today, the reality is very different. The network has emerged as the most reliable component.

\section{Integration at the FileDownload Level}

\subsection{Changes to the FileDownload agent}

\subsection{Setup and test}
The prototype was tested on ANSE's testbed (Figure \ref{fig:testbed}) using 
two sites: T2\_ANSE\_Amsterdam and T2\_ANSE\_Geneva. These sites consist of two
 servers each: a PhEDEx server and a dedicated storage server for that particular 
 PhEDEx instance. Each of the two storage servers consist of dual 8 core CPUs 
 (with HT), 64GB of RAM and 2 LSI controllers which manage 8 SSDs each. We created 
 two RAID-0 partitions of 4 SSDs each on every LSI controller. Since the tests are write 
 intensive and were scheduled to run for around 24 hours, we decided to only use 
 1 controller, in order to minimize the negative effects on the lifetime of the disks.

The two sites were connected via a high speed 40Gbps link. On this link we could create
static virtual circuits and for the purposes of our test we decided to create two new
circuits of 10 Gbps each.

One of these circuits was used to model a shared link in which PhEDEx had to compete 
with other traffic. This background traffic was generated by Iperf and consisted of a 
continuous stream of UDP packets at 5Gbps . The other circuit served as the dedicated
link.

Monitoring was done with MonALISA\footnote{MONitoring Agents using Large
 Integrated Services Architecture (monalisa.cern.ch) } and the PhEDEx tool itself.

The main purpose of this test wasn't to show that we can saturate a 10Gbps link with
PhEDEX (although we came close with just 1 controller), but that a PhEDEx instance 
is able to switch to using a new path in a transparent manner for the other instances
and with no down time. 

The first part of the test consisted of a 10 hour run with PhEDEx transfers on the 
shared link. After this time, PhEDEx switched to using the dedicated circuit and
continued transfers for another 10 hours.

We set up PhEDEx to run a single 450 GB transfer job at a time, each one 
comprising of 30 files of 15 GB each. 

\begin{figure}[h]
  \centering
  \includegraphics[width=0.95\textwidth]{Figures/FileDownload_ANSE_Testbed}
  \caption{Diagram of the testbed used by the prototype}
  \label{fig:testbed}
\end{figure} 


\subsection{Results}

The results of the first half of the test are shown in  (Figure \ref{fig:shared_transfers}).
A quick glance at this plot, shows that the 10 Gbps link was saturated by the two
competing transfers. Between 23:00 and 0:00 (beginning of the plot) and 
10:00 and 11:00 (end of the plot), only UDP traffic was sent across the network, 
running at a steady 5 Gbps. This effectively leaves only 5 Gbps of PhEDEx traffic.
PhEDEx transfers start around 0:00 and quickly saturate the 10 Gbps. 
The seesaw look of this plot is due to the fact that PhEDEx has a delay
between finishing one job and starting the next. This is due to various factors:
 pre/post validation, preparation of copyjobs or even time spent by the backend itself 
 before actually launching a transfer. Because of these delays, the average 
 rates reported by PhEDEx will always be lower than the average rates of each
 individual transfer job. Nevertheless we get average transfer rates of 9.5 Gbps for the
 whole link and consequently 4.5 Gbps for PhEDEx transfers (10\% penalty from
 gap between jobs).

\begin{figure}[h]
  \centering
  \includegraphics[width=0.95\textwidth]{Figures/FileDownload_Shared_path.png}
  \caption{PhEDEx transfers on the shared path - competing with 5Gbps UDP traffic}
  \label{fig:shared_transfers}
\end{figure} 

Around 10:00 PhEDEx switches to using the dedicated link. The results of the 
second half of the test are shown in Figure \ref{fig:solo_transfers}.

Although most of the time we are able to saturate the 10Gbps link with PhEDEx 
traffic alone we sometimes see dips in transfer rates. These dips can be attributed 
to the storage not being able to sustain such high write rates.

This time, the average link rates drop from 9.5 Gbps to 8.5 Gbps, however 
actual PhEDEx transfers go up from 4.5 Gbps to 8.5 Gbps! The reason for this 
drop in average link rate is two fold: 

\begin{itemize}
  \item Transferring a job at higher speeds means that it will take less time for
  it to complete, hence more jobs will be completed in one hour. However, as
  we previously explained, there is a delay before starting each new job, which
  means that more delays will be introduced into the system.
  \item The storage system (using 1 LSI controller) sometimes cannot sustain
  write rates of 10 Gbps to disk.
\end{itemize}


\begin{figure}[h]
  \centering
  \includegraphics[width=0.95\textwidth]{Figures/FileDownload_Solo_path.png}
  \caption{PhEDEx transfers on the dedicated path}
    \label{fig:solo_transfers}
\end{figure} 

Figure \ref{fig:combined_transfers} and Figure \ref{fig:combined_phedex_transfers} show
the  network activity for PhEDEx traffic alone on both links. 
In this plot we can observe that PhEDEx switches to using the
new link without any interruption in service, doing it seamlessly and in a 
transparent way for the other instances (switch occurs around 10:00).

This plot also shows that there is a huge benefit to using the dedicated link.

\begin{figure}[h]
  \centering
  \includegraphics[width=0.95\textwidth]{Figures/FileDownload_All_paths.png}
  \caption{View of PhEDEx only transfers on both the shared and dedicated path}
  \label{fig:combined_transfers}
\end{figure} 

\begin{figure}[h]
  \centering
  \includegraphics[width=0.95\textwidth]{Figures/FileDownload_PhEDEx_all_paths.png}
  \caption{View of PhEDEx only transfers on both the shared and dedicated path}
  \label{fig:combined_phedex_transfers}
\end{figure} 
%% Vlad, this is for you
\begin{thebibliography}{1}

\bibitem{PhEDEx}
 Egeland R, Metson S and Wildish T 2008 Data transfer infrastructure for CMS data taking,  {\it XII Advanced Computing and Analysis Techniques in Physics Research (Erice, Italy: Proceedings of Science)}

\bibitem{CMS}
 The CMS Collaboration 2008 The CMS experiment at the CERN LHC {\it JINST {\bf 3} S08004}

\bibitem{FTS}
 Alvarez Ayllon A, Kamil Simon M, Keeble O and Salichos M 2013 FTS3 - Robust, simplified and high-performance data movement service for WLCG {\it submitted to CHEP 2013}

\bibitem{MonALISA}
 Legrand I, Newman H, Voicu R, Cirstoiu C, Grigoras C, Dobre C, Muraru A, Costan A, Dediu M and Stratan C 2009 MonALISA: An agent based, dynamic service system to monitor, control and optimize distributed systems {\it Computer Physics Communications, Volume 180, Issue 12, December 2009, Pages 2472 - 2498}

\bibitem{MonALISA site}
 MONitoring Agents using a Large Integrated Services Architecture (MonALISA) {\it http://monalisa.caltech.edu/}
 
\bibitem{FDT}
 Fast Data Transfer (FDT) {\it http://fdt.cern.ch/}
 
\bibitem{WLCG}
 Eck C {\it et al.} 2005 LHC Computing Grid Technical Design Report {\it CERN-LHCC-2005-024}

\bibitem{TW_DB_CHEP13}
 Bonacorsi D and Wildish T 2013 Challenging CMS Computing with Network-Aware Systems {\it submitted to CHEP 2013}

\bibitem{ANSE}
 \textbf{ANSE: Need a reference}

\bibitem{perfSONAR}
  Campana S et.al. 2013 Deployment of a WLCG network monitoring infrastructure based on the perfSONAR-PS technology {\it submitted to CHEP 2013}

\bibitem{OSCARS}
  Guok C, Robertson D, Thompson M, Lee J, Tierney B, Johnston W 2006: Intra and Interdomain Circuit Provisioning Using the OSCARS Reservation System {\it 3rd International Conference on Broadband Communications, Networks and Systems, 2006. BROADNETS 2006}

\bibitem{POE}
 The Perl Object Environment (POE) {\it http://poe.perl.org/}

% F.~Baggins, \emph{Quantum effects of the One Ring},
% \emph{JHEP} {\bf 01} (3021) 006 [{\tt hep-th/2001033}].

\end{thebibliography}


\end{document}