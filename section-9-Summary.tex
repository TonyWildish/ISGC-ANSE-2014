\section{Summary and future plans}

PhEDEx has been very successful at managing data-flows on the WAN for the CMS collaboration. Nonetheless, it's architecture is based on design decisions that are no longer valid, and in order to continue to scale and perform for the future, it must evolve, taking advantage of new technologies.

Over the course of the past year, the ANSE project has made significant progress
towards integrating network-awareness into PhEDEx. We have
created and tested a prototype which integrates the use of virtual-circuits into
PhEDEx at the level of the FileDownload agent, i.e. per destination-site. This is transparent for the other site-agents, with no modifications to the database schema and with all the logic contained in the FileDownload
agent. Nor does it depend on network-circuits being infallible, the existing failure and retry mechanisms can cope with any underlying instability in the network.

Our immediate goal is to adopt a circuit management system and use its API to
control and operate real circuits. This will be followed by extensive tests, on 
a potentially expanded ANSE testbed and ultimately in production data-transfers.

The next step is to integrate the use
of virtual circuits at the FileRouter level. Here, we gain a global view of 
CMS-wide flows, which means we can make informed (optimal) decisions about which 
circuits need to be requested or not.