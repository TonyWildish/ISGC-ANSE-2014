\section{Integrating virtual circuts into PhEDEx}
There are several points in the PhEDEx software stack at which it is possible to integrate the control and use of virtual circuits; per transfer-job, at the level of the FileDownload agent, and at the FileRouter agent.

At the lowest level, a circuit can be requested per transfer-job, by the transfer tool called from the FileDownload agent. This level of integration was already achieved some time ago with the initial implementation of the FDT backend for the FileDownload agent. Each transfer-job would attempt to create its own virtual circuit, if it failed it would simply proceed to transfer over the general-purpose network.

The advantage of this approach is simplicity, there is nothing to modify in PhEDEx itself, all the work happens in the external transfer tool. There are, however, a few disadvantages:
\begin{itemize}
\item Transfer-jobs tend to be short-lived (tens of minutes rather than hours), so circuits are being created at a high rate. This is not currently sustainable.
\item PhEDEx is typically configured to overlap transfer-jobs, to offset the overheads incurred between successive jobs. This means that concurrent transfer-jobs will be competing for circuits, with consequences that may be hard to understand. For example, if two jobs can share a circuit, what happens to one of them when the other finishes, and tears down the circuit?
\item There is no overview of the use of network resources. Circuits are requested based on nothing more than the local knowledge of a single link, without regard for other traffic flows into the destination or out of the source.
\end{itemize}

The next level of integration is at the level of the FileDownload agent. This site-agent has an overview of the data-flows into the destination site at any time, so can make intelligent decisions about creating circuits. Apart from simply deciding that it should - or should not - create a circuit for a given flow, it can also create circuits that are longer-lived, serving a series of transfer-jobs lasting several hours. This will make for much more stable operation, both from the viewpoint of the network fabric and from the viewpoint of the transfer-jobs. This is the first level of integration at which virtual circuit control becomes really practical for production use by PhEDEx. The only significant disadvantage is that destination sites still make circuit-requests without regard for conditions at the source sites, so the risk of sub-optimal configurations still exists. For example, site A may book a circuit to site B to pull data from it at high speed, but that may commit most of site B's WAN bandwidth, leaving little for site B to download from sites C, D and E.

The highest level of integration is at the level of the FileRouter, or centrally, with a view of the entire transfer system. The FileRouter is essentially a high-level scheduler, deciding the set of source-destination pairs for fulfilling transfer requests and building the queues of transfers for each destination site. As such, it alone can have an overview of how busy each site is expected to be, so it could decide which transfers should use circuits and which should not. It could even base its choice of source-replica for a given transfer on availability of circuits and bandwidth, not just on current network performance between the sites. It can also decide that some transfers are low priority, so should not request a circuit even if one might be available, in order to leave that resource free for higher priority requests that might come along.

There are two disadvantages to integration at the FileRouter. The first is that the FileRouter deals with transfers that \emph{will} happen, in the near future. When something happens to invalidate its predictions (e.g. a power-cut at a site) this will invalidate those predictions to a greater or lesser degree. The more coupled the sites are in their scheduling, the wider the implications of a single site-outage.

The second disadvantage is the complexity of the problem. Developing efficient scheduling algorithms, including predicting the availability of circuits on a shared network, is not a trivial task. For example, queues may be re-organised to take advantage of circuits, instead of just using raw priority and age as the queueing criteria. The FileRouter agent will also have to inform FileDownload agents when they are expected to use a circuit and when not to, and not just which files to transfer. Despite these complications, having centrally managed knowledge of when and where to create circuits for PhEDEx is the ultimate goal.